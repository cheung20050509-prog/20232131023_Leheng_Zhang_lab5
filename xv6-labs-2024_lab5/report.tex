\documentclass[UTF8,a4paper,12pt]{article}
\usepackage{ctex}
\usepackage{graphicx}
\usepackage{geometry}
\usepackage{listings}
\usepackage{xcolor}
\usepackage{float}
\usepackage{hyperref}

\geometry{left=2.5cm,right=2.5cm,top=2.5cm,bottom=2.5cm}

\title{\textbf{操作系统课程设计 Lab 5: Network Driver 实验报告}}
\author{Jacob}
\date{\today}

\begin{document}

\maketitle

\section{实验目的}
本次实验旨在为 xv6 操作系统添加网络功能。主要任务包括编写 Intel E1000 网卡驱动程序(包含发送和接收功能),并在此基础上实现一个简单的 UDP 网络协议栈,支持 socket 接口(bind, recv),从而使 xv6 能够与宿主机进行网络通信。

\section{实验内容与实现}

\subsection{Part 1: 网卡驱动 (NIC Driver)}
实验的第一部分是让 xv6 能够通过 E1000 网卡驱动收发数据包。
x-special/nautilus-clipboard copy file:///tmp/VMwareDnD/id3yE6/task1\_flowchart.
png
\subsubsection{1. 发送功能的实现 (e1000\_transmit)}
我们实现了 \texttt{e1000\_transmit} 函数。该函数将上层协议栈(ARP/IP)传递下来的 socket buffer (mbuf) 放入发送环形缓冲区(TX Ring)中。
x-special/nautilus-clipboard copy file:///tmp/VMwareDnD/id3yE6/task1\_flowchart.
png
关键步骤包括:
\begin{enumerate}
    \item 读取 \texttt{E1000\_TDT} 寄存器获取下一个可用的描述符索引。
    \item 检查该描述符的 \texttt{E1000\_TXD\_STAT\_DD} 标志位,确认上一轮发送是否完成。
    \item 释放旧的缓冲区(如果有),并将新数据的物理地址填入描述符。
    \item 设置 CMD 标志位(EOP 和 RS)。
    \item 更新 \texttt{E1000\_TDT} 寄存器通知网卡开始工作。
\end{enumerate}

\begin{figure}[H]
    \centering
    \includegraphics[width=0.8\textwidth]{../txone_ok.png}
    \caption{e1000\_transmit 功能测试通过 (txone)}
    \label{fig:txone}
\end{figure}

\subsubsection{2. 接收功能的实现 (e1000\_recv)}
我们实现了 \texttt{e1000\_recv} 函数。该函数轮询接收环形缓冲区(RX Ring),提取网卡已填充数据包的描述符。
关键步骤包括:
\begin{enumerate}
    \item 读取 \texttt{E1000\_RDT} 并加一取模,获取下一个待处理的描述符位置。
    \item 检查 \texttt{E1000\_RXD\_STAT\_DD} 标志位,确认是否有新包到达。
    \item 将缓冲区数据通过 \texttt{net\_rx} 传递给协议栈。
    \item 分配新的空闲缓冲区(kalloc)替换已被取用的缓冲区,重置描述符状态。
    \item 更新 \texttt{E1000\_RDT} 寄存器。
\end{enumerate}

\begin{figure}[H]
    \centering
    \includegraphics[width=0.8\textwidth]{../rxone.jpg}
    \caption{e1000\_recv 功能测试通过 (rxone: arp\_rx/ip\_rx)}
    \label{fig:rxone}
\end{figure}

\subsection{Part 2: UDP套接字 (UDP Sockets)}
实验的第二部分是实现用户态网络接口。

\subsubsection{1. 系统调用实现}
\begin{itemize}
    \item \textbf{sys\_bind}: 将 Socket 绑定到指定的本地端口。如果端口已被占用则返回错误。
    \item \textbf{sys\_recv}: 从 Socket 的接收队列中读取数据。如果队列为空,进程进入睡眠状态(sleep),直到被 \texttt{ip\_rx} 唤醒。
\end{itemize}

\subsubsection{2. 协议栈处理 (ip\_rx)}
修改了 \texttt{net.c} 中的 \texttt{ip\_rx} 函数。当收到 UDP 包时,根据目的端口号查找对应的 Socket,将数据包挂入该 Socket 的接收队列,并调用 \texttt{wakeup} 唤醒等待的进程。

\begin{figure}[H]
    \centering
    \includegraphics[width=0.8\textwidth]{../grade.jpg}
    \caption{完整测试通过 (Make Grade Score: 171/171)}
    \label{fig:grade}
\end{figure}

\section{实质性物证}

\subsection{项目仓库 (Repository)}
完整的项目代码及历史提交记录已上传至 GitHub:
\begin{itemize}
    \item \url{https://github.com/cheung20050509-prog/20232131023_Leheng_Zhang_lab5}
\end{itemize}

\subsection{调试日志 (Debug Log)}
\begin{itemize}
    \item \textbf{初始环境搭建}: 切换到 \texttt{net} 分支,确认 \texttt{make qemu} 环境正常。阅读 \texttt{e1000\_dev.h} 熟悉寄存器定义。
    \item \textbf{发送功能调试}:
        \begin{itemize}
            \item \textbf{问题}: 第一次实现时,忘记检查 TX Ring 是否已满(DD标志),导致覆盖未发送数据。
            \item \textbf{解决}: 增加 \texttt{status \& E1000\_TXD\_STAT\_DD} 检查逻辑。
            \item \textbf{问题}: 主机端收不到包。
            \item \textbf{解决}: 发现未添加 \texttt{\_\_sync\_synchronize()} 内存屏障,添加后修复。
        \end{itemize}
    \item \textbf{接收功能调试}:
        \begin{itemize}
            \item \textbf{问题}: 接收包后数据错乱。
            \item \textbf{解决}: 发现复用了同一个 mbuf,修改为每次接收后立即 \texttt{kalloc} 新的缓冲区填入 Ring。
        \end{itemize}
    \item \textbf{UDP/DNS 调试}:
        \begin{itemize}
            \item \textbf{问题}: \texttt{make grade} 中 DNS 测试失败。
            \item \textbf{排查}: 通过 printf 调试发现 payload 长度不对。
            \item \textbf{解决}: 修正了 \texttt{sys\_recv} 和 \texttt{ip\_rx} 中关于 IP 头长度和 UDP 包长度的计算公式。
        \end{itemize}
\end{itemize}

\subsection{用户手册 (User Manual)}
\subsubsection{功能特性}
本系统扩展了 xv6 内核,支持 E1000 网卡驱动及 UDP/IP 协议栈。

\subsubsection{API 接口}
\begin{lstlisting}[language=C, basicstyle=\small\ttfamily, frame=single]
// 1. 绑定端口
// 成功返回 0,失败返回 -1
int bind(short port);

// 2. 发送数据
// dst 为目的IP,dport 为目的端口
int send(short sport, int dst, short dport, char *buf, int len);

// 3. 接收数据
// 阻塞直到收到数据。src/sport 将填入发送方信息。
int recv(short dport, int *src, short *sport, char *buf, int maxlen);
\end{lstlisting}

\subsubsection{使用方法}
\begin{verbatim}
$ make qemu
$ nettest txone  # 测试发送
$ nettest rxone  # 测试接收 (需配合 server)
\end{verbatim}

\section{实验心得与 AI 使用感想}
本次实验是我在操作系统课程中最具挑战性的一次。通过亲手实现网卡驱动,我深刻理解了操作系统内核如何通过 DMA descriptor ring 与外设硬件进行交互,以及软硬件协同工作的奥妙。这也让我明白了为什么网络 I/O 通常是异步的,以及操作系统如何通过中断和轮询的结合来高效处理数据流。

在本次实验中,我借助了 \textbf{GitHub Copilot} 等大模型工具辅助开发,其作用主要体现在以下几个方面:

\begin{enumerate}
    \item \textbf{文档阅读与概念解析}: E1000 的硬件手册非常厚重,寄存器繁多。通过询问大模型,我能够快速定位到 \texttt{TDT}, \texttt{RDT} 等关键寄存器的作用,以及 \texttt{TX/RX Descriptor} 中各个标志位的含义(如 \texttt{CMD\_EOP}, \texttt{STAT\_DD}),这极大地缩短了我的“冷启动”时间。
    \item \textbf{代码辅助与纠错}: 在编写驱动代码时,AI 能够根据上下文补全冗长的结构体定义和宏定义代码。更重要的是,它像一位结对编程的导师,在我忘记编写内存屏障(Memory Barrier)或忘记并在临界区加锁时,给出了风险提示,帮助我规避了潜在的并发 Bug。
    \item \textbf{调试思路启发}: 在遇到 DNS 测试不通过的问题时,我将现象描述给 AI,它提示我关注“数据包长度计算”和“字节序转换”问题。这一提示直接帮助我发现了代码中未正确减去 UDP 头部长度的逻辑错误。
\end{enumerate}

总的来说,AI 并没有替代我的思考,而是成为了放大我能力的工具,让我从繁琐的语法和文档检索中解放出来,能够更专注于操作系统核心逻辑(如并发控制、内存管理)的思考与实现。

\end{document}
